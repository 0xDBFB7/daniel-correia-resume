\documentclass[fleqn,11pt]{article}

\usepackage[left=2cm,right=2cm,
top=1.25cm,
bottom=0.5cm,%
headheight=11pt,%
letterpaper]{geometry}

\include{preamble}


\begin{document}




Daniel Correia\\
30 Lamont Creek Drive\\
Wasaga Beach, Ontario\\
L9Z 1J9\\



\noindent\par
\noindent\makebox[\textwidth][c]{%
\begin{minipage}{0.8\textwidth}

Dear Recruiting Manager,\\


Hi!\\

Factors which decrease my suitability for the position of \textit{Equipment Technician, Epitaxy}:\\

\begin{itemize} 
	\item No experience with any semiconductor process-specific chemistry or safety beyond inert gases
	\item No experience maintaining vacuum cleanliness below $10^{-6}$ mbar.\\
\end{itemize}

However, I have:\\

\begin{itemize}
\item Hands-on experience fabricating and maintaining high-vacuum equipment (turbo and diff. pumps, ion gauges) as a side project,
\item A BSc in Science,
\item Experience independently managing and conducting work,
\item Extensive experience with electronics design, prototyping and repair; instrumentation; software development and simulation; and literature review.\\
\end{itemize}

Thanks for your time!\\



\end{minipage}}
%\newgeometry{textwidth=10cm,textheight=10cm}



\restoregeometry


\clearpage









\begingroup
\fontseries{t}\selectfont

{\Huge Hi, I'm Daniel.}

\endgroup

\light{\large \textit{Science rules!}}

%{
%\centering
\small{{Daniel Correia}\ \orcidlink{0000-0002-9353-0216} | \href{https://github.com/0xDBFB7}{github.com/0xDBFB7} | therobotist+resume@gmail.com | @0xDBFB7 | 1-705-606-8866\\
%}
\light{\makebox[\linewidth]{\rule{\textwidth}{0.4pt}}}
% dcorreia@whimsysciences.com

\begin{multicols}{1}

\begin{tcolorbox}
\textbf{Education:\\}
B.Sc. in Science from York University. Graduated January 2021.
\end{tcolorbox}

\subsection*{SafeSump Inc.}

Founder/CTO of four-year project to design and produce a failure-resistant water pump system. Funded by:
%
\begin{itemize}
\item \$37,500 Ontario Centres of Excellence  \href{https://drive.google.com/file/d/1WXrxVwTggaL7WEvLv6DgJ891fSo7LqqP/view?usp=sharing}{grant \#26828} (2016-2020)
\item \$75,000 government \href{https://drive.google.com/file/d/11pdJNzYDE-28X3m0rH8mE4cxoliTJZGH/view?usp=sharing}{contract} through Federation of Canadian Municipalities (2018-2020).
\end{itemize}
Broad overview of skills gained:

\begin{itemize}
	\item \textbf{Electronics}: Custom hardware development from in-house prototyping to volume production; design of ultrasonic and capacitive sensors
	\item \textbf{Software and firmware}: Writing and maintaining a 20k SLOC codebase of Python and embedded C and C++ on STM32 and SAM devices. Version control, frontend and backend programming; Linux administration.
	\item \textbf{Soft skills}: Pair programming, management, conducting presentations, collaboration
\end{itemize}

\subsection*{Kesti Engineering Ltd.}

Occasional board-level repair on Mazak and Haas CNC machines. A successful repair documented \href{https://0xdbfb7.com/meldas.html}{here}.
%\subsection*{Personal projects}
%\subsection*{\textit{Musings on an inexpensive 1500\textdegree C silicon carbide furnace}}
%
%An attempt to make various high-performance ceramic techniques available to a broader audience.\\ Built on the work of more than 200 scientific papers.
%
%A short report on some parts can be found at \href{https://0xdbfb7.com/furnace.html}{0xdbfb7.com/furnace.html}.
%
\subsection*{Vacuum systems}

\href{https://github.com/0xDBFB7/ionprinter/}{github.com/0xDBFB7/ionprinter/}: A multi-year attempt to explore high-current ion beam lithography. 

Spinoff projects:
\begin{itemize}
	\item a GPU-accelerated Particle-In-Cell program called \href{https://github.com/0xDBFB7/Nyion}{Nyion}
	\item An inexpensive silicon carbide \href{https://0xdbfb7.com/furnace.html}{furnace}:\\ \\
	An attempt to make various high-performance ceramic techniques available to a broader audience.\\ Built on the work of more than 200 scientific papers. 
	\item An \href{https://github.com/0xDBFB7/varian-turbo-controller}{inexpensive aftermarket controller} for Varian Turbo-V200 series of turbomolecular pumps 
	\item \href{https://gist.github.com/0xDBFB7/7bd7048c6639270e6f291a2673903184}{Control software} for an Inficon BPG-400 vacuum gauge
\end{itemize}

\subsection*{\textit{Some notes and experiments on the electropermeabilization of viral membranes}}

Technical report at doi:\href{https://doi.org/10.5281/zenodo.4568507}{10.5281/zenodo.4568507}



	
%
%\subsection{Broad overview of skills gained from these projects}
%\begin{itemize}
%	\item \textbf{Documentation}: LaTeX, Jupyter notebooks, Reference management
%	\item \textbf{Software}: Data analysis and automation; Python, C++, Mathematica, with a smattering of Julia and MATLAB.
%	\item \textbf{Simulation}: Several dozen toolchains were in use, ranging from modified open-source electromagnetic simulation systems to molecular dynamics with GROMACS.
%	\item \textbf{Electronics}: Microwave electronics design, PCB design with KiCAD
%	\item \textbf{Fabrication}: Electronics prototyping, CNC mill and lathe operation, micromachining, microfluidics
%\end{itemize}


%
\begin{tcolorbox}
	\textbf{Personal:\\}
	Member of SimCoLab Barrie (now BRiX) hackerspace for 7 years.\\
	Canadian and German citizenship.
\end{tcolorbox}
%



\end{multicols}


\section{Gallery}
\begin{figure}[H]
	%	\makebox[\textwidth][c]{
	\centering
	\subfloat[]{
		\includegraphics[width=0.5\textwidth]{vacuum}
	}
	\subfloat[]{
		\includegraphics[width=0.5\textwidth]{digraph.png}
	}
	\caption*{A bespoke high-vacuum system. GPU-accelerated multigrid data structure and electrostatics solver for particle-in-cell ion beam simulation.}
	\hfill
	
\end{figure}



\begin{figure}[H]
	%	\makebox[\textwidth][c]{
	\centering
	\subfloat[]{
		\includegraphics[width=0.4\textwidth]{image1}
		
	}
	\subfloat[]{
		\includegraphics[width=0.4\textwidth]{safesump.png}
		
	}
	\caption*{Redundant controller designed for SafeSump Inc.}
	\hfill
	
\end{figure}



\begin{figure}[H]
	\centering
	
	\subfloat[]{
		\includegraphics[width=0.5\textwidth]{pulse_exposure_setup}
		
	}
	\subfloat[]{
		\includegraphics[width=0.5\textwidth]{eppenwolf_2}
	}
	
	\subfloat[]{
		\includegraphics[width=0.4\textwidth]{x_gal.jpg}
	}
	\hfill
	\subfloat[]{
		\includegraphics[width=0.5\textwidth]{bronch_9GHz_500W_2}
	}
	
	\caption*{\\ (a) A sub-nanosecond kilovolt pulse exposure cell (based on an off-the-shelf avalanche transistor pulser design). \\ (b) A 12 GHz microwave absorption spectrometer. \\(c) The very pretty opalescent blue culture caused by E. coli B trying to metabolize lactose in an indicator for the enzyme $\beta$-galactosidase.\\ (d) An FDTD simulation of electromagnetic interaction with tissue. }
\end{figure}







\end{document}