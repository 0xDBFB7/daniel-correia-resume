\documentclass[fleqn,10pt]{article}

\usepackage[left=2cm,right=2cm,
top=1.25cm,
bottom=0.5cm,%
headheight=11pt,%
letterpaper]{geometry}

\include{preamble}

\begin{document}

%Daniel Correia\\
%30 Lamont Creek Drive\\
%Wasaga Beach, Ontario\\
%L9Z 1J9
%
%Canadian Light Source Inc.\\
%44 Innovation Boulevard\\
%Saskatoon, SK S7N 2V3\\
%Canada\\
%
%\noindent\par
%\noindent\makebox[\textwidth][c]{%
%\begin{minipage}{0.7\textwidth}
%Dear Recruiting Manager,\\
%
%Hello!\\
%
%In response to SXRMB Support Scientist position number 2021-2, kindly find my resume attached. \\
%
%I have about 5 years of experience founding and managing an electronics company and performing associated research, along with a B.Sc. in Science. \\\\ Skills particularly suited to the CLS include some experience in the design and fabrication of high vacuum systems, and with beam simulation codes such as IBSimu and Warp.\\
%
%I also have a small amount of recent experience in physical virology and applied microbiological technique that may be of some use.\\
%
%
%
%I am willing to relocate for this position, and am available for a remote interview at any time. \\\\
%
%Thanks for your time!
%\end{minipage}}
%%\newgeometry{textwidth=10cm,textheight=10cm}
%
%
%
%%\restoregeometry
%
%
%\clearpage

\begingroup
\fontseries{t}\selectfont

{\Huge Hi, I'm Daniel.}

\endgroup

\light{\large \textit{Science rules!}}



\small{{Daniel Correia}\ \orcidlink{0000-0002-9353-0216} | \href{https://github.com/0xDBFB7}{github.com/0xDBFB7} | therobotist@gmail.com ({\footnotesize preferred}) | dcorreia@safesump.com | @0xDBFB7 on Twitter
\light{\makebox[\linewidth]{\rule{\textwidth}{0.4pt}}}


\begin{multicols}{1}

\begin{tcolorbox}
\textbf{Education:\\}
B.Sc. in Science from York University, Physics stream. Graduated in the fall of 2021.
\end{tcolorbox}
%
\section*{SafeSump Inc.}

CEO/CTO of four-year project to design and produce a failure-resistant water pump system. Funded by a \$37,500 Ontario Centres of Excellence grant (2017-2020) followed by a \$75,000 government contract (2018-2020).

Broad overview of skills gained:

\begin{itemize}
	\item \textbf{Electronics}: Production electronics design and design-for-manufacturing; rapid prototyping of ultrasonic and capacitive sensors, among others
	\item \textbf{Software and firmware}: Version control. Frontend and backend server programming; Linux administration, programming of production utilities and scripts. Writing and maintaining a 30k SLOC codebase of C, Python and C++.
	\item \textbf{Soft skills}: Pair programming, time management, writing.
\end{itemize}


\section*{Viral electroporation}

A 10-month attempt to follow up experimentally on previous research regarding the dielectric properties of viruses, in the hopes of harnessing a phenomenon known as irreversible membrane electroporation, via optimized Brillouin precursors.

This required some literature review and a degree of care in experimental design, but also required several specialized pieces of equipment:

A custom, inexpensive synchronous photon-counting fluorescence system, allowing amplification-free quantification of dsDNA at sub-nanogram resolution.

Development of an inexpensive 12 GHz microwave absorption spectrometer (albeit unused in the final experiment)

Electromagnetic modelling of tissue and experimental system parameters via FDTD; sub-nanosecond kilovolt pulse generation; numerical optimization of dispersive pulses.

Data was collected on the model organism bacteriophage T4, necessitating wet-lab techniques.\\ \\ Due to several missteps in design on my part, however, the study was thoroughly inconclusive. A report on the same is due to be edited.


Broad overview of skills gained:
\begin{itemize}
	\item \textbf{Documentation}: LaTeX, Jupyter notebooks, Mathematica, Reference management
	\item \textbf{Software}: Data analysis and automation; Python, C++, and a smattering of many others
	\item \textbf{Simulation}: Several dozen toolchains were in use, ranging from modified open-source electromagnetic simulation systems to molecular dynamics with GROMACS.
	\item \textbf{Electronics}: Microwave electronics design, PCB design with KiCAD
	\item \textbf{Fabrication}: Electronics prototyping, CNC mill and lathe operation, micromachining, microfluidics
\end{itemize}


\section*{Vacuum systems}

A 2-year attempt to develop a high-current ion beam lithography system, involving the ion-beam simulation tools noted in the cover letter and several custom solvers.
	
\end{multicols}



\begin{figure}[H]
	\centering
	
	\subfloat[]{
		\includegraphics[width=0.5\textwidth]{pulse_exposure_setup}
		
	}
	\subfloat[]{
		\includegraphics[width=0.5\textwidth]{eppenwolf_2}
	}
	
	\subfloat[]{
		\includegraphics[width=0.4\textwidth]{x_gal.jpg}
	}
	\hfill
	\subfloat[]{
		\includegraphics[width=0.5\textwidth]{bronch_9GHz_500W_2}
	}
	
	\caption*{\\ (a) The sub-nanosecond pulse generator and microfluidic exposure cell designed to induce electroporation. \\ (b) A 12 GHz microwave absorption spectrometer. \\(c) The very pretty opalescent blue culture caused by E. coli B trying to metabolize lactose in an indicator for the enzyme $\beta$-galactosidase.\\ (d) An FDTD simulation of electromagnetic interaction with tissue. }
\end{figure}




\begin{figure}[H]
	%	\makebox[\textwidth][c]{
	\centering
	\subfloat[]{
		\includegraphics[width=0.5\textwidth]{vacuum}
	}
	\subfloat[]{
	\includegraphics[width=0.5\textwidth]{digraph.png}
	}
	\caption*{Bespoke high vacuum system. GPU-accelerated multigrid data structure and electrostatics solver for particle-in-cell ion beam simulation}
	\hfill
	
\end{figure}



\begin{figure}[H]
	%	\makebox[\textwidth][c]{
	\centering
	\subfloat[]{
		\includegraphics[width=0.4\textwidth]{image1}
		
	}
	\subfloat[]{
		\includegraphics[width=0.4\textwidth]{safesump.png}
		
	}
\caption{Redundant controller with a 120 Mhz Atmel ARM processor, running \ntilde10k lines of high-reliability firmware.}
	\hfill
	
\end{figure}




\end{document}