\documentclass[fleqn,11pt]{article}

\usepackage[left=2cm,right=2cm,
top=1.25cm,
bottom=0.5cm,%
headheight=11pt,%
letterpaper]{geometry}

\include{preamble}

\usepackage{xhfill}


\newcommand{\ressection}[1]{\textbf{{\Large \textit{#1}}}\xrfill[0.1ex]{0.6pt}}

\newcommand{\sk}[1]{\textcolor{orange}{#1}}
\newcommand{\tec}[1]{\textcolor{olive}{\textbf{#1}}}
\begin{document}




Daniel Correia\\
30 Lamont Creek Drive\\
Wasaga Beach, Ontario\\
L9Z 1J9\\



\noindent\par
\noindent\makebox[\textwidth][c]{%
\begin{minipage}{0.8\textwidth}

Dear Recruiting Manager,\\


Hi!\\

Factors which decrease my suitability for the position of \textit{Equipment Technician, Epitaxy}:\\

\begin{itemize} 
	\item No experience with any semiconductor process-specific chemistry or safety beyond inert gases
	\item No experience maintaining vacuum cleanliness below $10^{-6}$ mbar.\\
\end{itemize}

However, I have:\\

\begin{itemize}
\item Hands-on experience fabricating and maintaining high-vacuum equipment (turbo and diff. pumps, ion gauges) as a side project,
\item A BSc in Science,
\item Experience independently managing and conducting work,
\item Extensive experience with electronics design, prototyping and repair; instrumentation; software development and simulation; and literature review.\\
\end{itemize}

Thanks for your time!\\



\end{minipage}}
%\newgeometry{textwidth=10cm,textheight=10cm}



\restoregeometry


\clearpage





% remember to rename file by name!



\begingroup
\fontseries{t}\selectfont

{\Huge Hi, I'm Daniel!}

\endgroup

\light{\large \textit{Science rules!}}

%{
%\centering
\begin{center}
\small{{Daniel Correia}\ \orcidlink{0000-0002-9353-0216} | \href{https://github.com/0xDBFB7}{github.com/0xDBFB7} | therobotist@gmail.com \textit{(preferred)} | @0xDBFB7 | 1-705-606-8866}\\
\light{\makebox[\linewidth]{\rule{\textwidth}{0.4pt}}}
\end{center}
% dcorreia@whimsysciences.com

% \begin{multicols}{1}

% 8621 https://learner.mycreds.ca/#/sharelink/b664abe7-53a7-4d64-a0f1-bef16337edd0/57724eeb-34ab-4b79-b2a6-cd6bc311039e

\begin{tcolorbox}
\textbf{Education:} B.Sc. in Science from York University. Graduated January 2021. \href{https://learner.mycreds.ca/#/sharelink/b664abe7-53a7-4d64-a0f1-bef16337edd0/57724eeb-34ab-4b79-b2a6-cd6bc311039e}{\textit{[verify diploma]}}
\end{tcolorbox}


\ressection{SafeSump Inc.}

Founder and CEO of four-year project to design and produce a failure-resistant water pump system to reduce flooding in homes.

Obtained \$37,500 \textit{Ontario Centres of Excellence} \href{https://drive.google.com/file/d/1WXrxVwTggaL7WEvLv6DgJ891fSo7LqqP/view?usp=sharing}{grant \#26828} (2016-2020) and \$75,000 \textit{Federation of Canadian Municipalities} \href{https://drive.google.com/file/d/11pdJNzYDE-28X3m0rH8mE4cxoliTJZGH/view?usp=sharing}{contract} (2018-2020) to develop and install systems in 50 homes.

I was responsible for: \\
- Developing backend data collection API (\sk{Python}, \sk{Linux}+\sk{Apache} stack, \sk{PostgreSQL}) and frontend UI\\
- Hardware design of embedded electronic controller (\sk{KiCAD} and \sk{Eagle}), FCC certification\\
- Conducting presentations\\
- Pair programming secure firmware (10k lines of \sk{embedded C} on STM32 and SAM4, \sk{C++}, \sk{make}, \sk{bash}), \\
- Developing data analysis, replication, filtering algorithms (\sk{Python}, \sk{gnuplot})\\
- Managing production processes, asset tracking, hardware-in-the-loop test systems, \\
- Series production of custom waterproof capacitive and \href{https://github.com/0xDBFB7/UltimateUltrasonicAmplifier}{ultrasonic} sensors,\\
- Version control (\sk{Git})
%\setlength{\columnseprule}{0.4pt}
%\columnbreak



\ressection{Research}

Independently designed and carried out an 18-month experiment in the field of bioelectrics on a tight budget, involving
\begin{itemize}\setlength\itemsep{-1em}
\item A custom 12 GHz microwave spectrometer and automated microfluidic sampling system
\item Custom \tec{FDTD} electromagnetic simulation software, \href{https://github.com/flaport/fdtd/pull/27}{now contributed upstream} 
\item Documentation in \sk{LaTeX}, \sk{Jupyter notebooks}, bibliography management with Zotero and Refbase
\item Visualization with Paraview and Chimera
\item Reviewing some 5,000 pages of scientific literature spanning 2,300 papers, highlighting 157 cited
\item A custom photomultiplier-based synchronous photon counting fluorescence system, capable of detecting single-digit nanograms of DNA (\sk{MAX10 and Cyclone IV FPGAs}, \sk{Verilog HDL}, \tec{photomultiplier tubes})
\item Molecular dynamics simulation with GROMACS
\item BSL-1 microbiology with E. coli B and T4 bacteriophage culture and plaque assays
\item Software-defined radio with HackRF SDR
\item Managed import paperwork, safety, and disposal of chemical reagents
\item Negotiating material transfer agreements for datasets
\item Symbolic mathematics with Maxima, SymPy, Mathematica, MATLAB Symbolic Toolbox
\end{itemize}

Preliminary technical report at doi:\href{https://doi.org/10.5281/zenodo.4568507}{10.5281/zenodo.4568507}
\pagebreak

\ressection{High vacuum systems}

\href{https://github.com/0xDBFB7/ionprinter/}{github.com/0xDBFB7/ionprinter/}: An exploration of high-current ion beam lithography techniques. 

Led to the following spinoff projects:
\begin{itemize}\setlength\itemsep{-1em}
	\item \href{https://github.com/0xDBFB7/Nyion}{Nyion}: a custom GPU-accelerated plasma simulation program using the particle-in-cell method on a custom block-structured mesh data structure for high computational efficiency\\
	(C++, cmake, OpenGL, gTest, HPC via CUDA, OpenCL, OpenMP 4.5 offload)
	\item An inexpensive silicon carbide \href{https://0xdbfb7.com/furnace.html}{furnace} capable of sintering oxide ceramics: an attempt to make various high-performance ceramic techniques available to a broader audience, built on the work of more than 200 scientific papers. 
	\item An \href{https://github.com/0xDBFB7/varian-turbo-controller}{inexpensive aftermarket controller} for the Varian Turbo-V series of turbomolecular pumps 
	\item \href{https://gist.github.com/0xDBFB7/7bd7048c6639270e6f291a2673903184}{Control software} for an Inficon BPG-400 vacuum gauge
\end{itemize}

\ressection{Kesti Engineering Ltd.}

Occasional board-level repair on Mazak and Haas CNC machines. A successful repair is documented \href{https://0xdbfb7.com/meldas.html}{here}.




	
%
%\subsection{Broad overview of skills gained from these projects}
%\begin{itemize}

%	\item \textbf{Software}: Data analysis and automation; Python, C++, Mathematica, with a smattering of Julia and MATLAB.
%	\item \textbf{Simulation}: Several dozen toolchains were in use, ranging from modified open-source electromagnetic simulation systems to molecular dynamics with GROMACS.
%	\item \textbf{Electronics}: Microwave electronics design, PCB design with KiCAD
%	\item \textbf{Fabrication}: Electronics prototyping, CNC mill and lathe operation, micromachining, microfluidics
%\end{itemize}


%
\begin{tcolorbox}
	\textbf{Personal:\\}
	Member of SimCoLab Barrie (now BRiX) hackerspace for 7 years.\\
	FIRST Robotics  Member of Stayner Team 2013	Cybergnomes	2010-2013\\
	Canadian and German citizenship.
\end{tcolorbox}
%


Video on electroadhesion 
https://www.youtube.com/watch?v=TQFxafFIoNM







\section{Gallery}
\begin{figure}[H]
	%	\makebox[\textwidth][c]{
	\centering
	\subfloat[]{
		\includegraphics[width=0.5\textwidth]{vacuum}
	}
	\subfloat[]{
		\includegraphics[width=0.5\textwidth]{digraph.png}
	}
	\caption*{A bespoke high-vacuum system. GPU-accelerated multigrid data structure and electrostatics solver for particle-in-cell ion beam simulation.}
	\hfill
	
\end{figure}



\begin{figure}[H]
	%	\makebox[\textwidth][c]{
	\centering
	\subfloat[]{
		\includegraphics[width=0.4\textwidth]{image1}
		
	}
	\subfloat[]{
		\includegraphics[width=0.4\textwidth]{safesump.png}
		
	}
	\caption*{Redundant controller designed for SafeSump Inc.}
	\hfill
	
\end{figure}



\begin{figure}[H]
	\centering
	
	\subfloat[]{
		\includegraphics[width=0.5\textwidth]{pulse_exposure_setup}
		
	}
	\subfloat[]{
		\includegraphics[width=0.5\textwidth]{eppenwolf_2}
	}
	
	\subfloat[]{
		\includegraphics[width=0.4\textwidth]{x_gal.jpg}
	}
	\hfill
	\subfloat[]{
		\includegraphics[width=0.5\textwidth]{bronch_9GHz_500W_2}
	}
	
	\caption*{\\ (a) A sub-nanosecond kilovolt pulse exposure cell (based on an off-the-shelf avalanche transistor pulser design). \\ (b) A 12 GHz microwave absorption spectrometer. \\(c) The very pretty opalescent blue culture caused by E. coli B trying to metabolize lactose in an indicator for the enzyme $\beta$-galactosidase.\\ (d) An FDTD simulation of electromagnetic interaction with tissue. }
\end{figure}







\end{document}