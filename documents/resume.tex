\documentclass[fleqn,11pt]{article}

\usepackage[left=2cm,right=2cm,
top=1.25cm,
bottom=0.5cm,%
headheight=11pt,%
letterpaper]{geometry}

\include{preamble}


\begin{document}
%
%Daniel Correia\\
%30 Lamont Creek Drive\\
%Wasaga Beach, Ontario\\
%L9Z 1J9\\


%
%\noindent\par
%\noindent\makebox[\textwidth][c]{%
%\begin{minipage}{0.7\textwidth}
%Dear Recruiting Manager,\\
%
%Hello!\\
%
%In response to Thermal-Mechanical Characterizations Facility Coordinator position, kindly find my resume attached. \\
%
%I have about 5 years of experience founding and managing an electronics company and performing associated research, along with a B.Sc. in Science. \\
%
%Several personal and company projects have involved high-temperature thermometry 
%
%I am willing to relocate for this position, and am available for a remote interview at any time. \\\\
%
%Please call at 705-606-8866 if you have any questions.\\
%
%Thanks for your time!
%\end{minipage}}
%%\newgeometry{textwidth=10cm,textheight=10cm}
%


%\restoregeometry
%
%
%\clearpage






%
%
%
%\begingroup
%\fontseries{t}\selectfont
%
%{\Huge Hi, I'm Daniel.}
%
%\endgroup
%
%\light{\large \textit{Science rules!}}
%
%%{
%%\centering
%\small{{Daniel Correia}\ \orcidlink{0000-0002-9353-0216} | \href{https://github.com/0xDBFB7}{github.com/0xDBFB7} | therobotist+resume@gmail.com | @0xDBFB7 | 1-705-606-8866\\
%%}
%\light{\makebox[\linewidth]{\rule{\textwidth}{0.4pt}}}
%
%
%\begin{multicols}{1}
%
%\begin{tcolorbox}
%\textbf{Education:\\}
%B.Sc. in Science from York University, with some focus on Physics. Graduated January 2021 with B+ GPA.
%\end{tcolorbox}
%
%\subsection*{SafeSump Inc.}
%
%Founder/CTO of four-year project to design and produce a failure-resistant water pump system. Funded by a \$37,500 Ontario Centres of Excellence grant (2017-2020) followed by a \$75,000 government contract (2018-2020).
%
%Broad overview of skills gained:
%
%\begin{itemize}
%	\item \textbf{Electronics}: Hardware development from in-house prototyping to volume production; design of ultrasonic and capacitive sensors
%	\item \textbf{Software and firmware}: Writing and maintaining a 20k SLOC codebase of Python and embedded C and C++ on STM32 and SAM devices. Version control, frontend and backend programming; Linux administration.
%	\item \textbf{Soft skills}: Pair programming, management, collaboration
%\end{itemize}
%
%\subsection{Personal research projects}
%
%\subsection*{\textit{Musings on an inexpensive 1500\textdegree C silicon carbide furnace}}
%
%An attempt to make various high-performance ceramic techniques available to a broader audience.\\ Built on the work of more than 200 scientific papers.
%
%A short report on some parts can be found at \href{https://0xdbfb7.com/furnace.html}{0xdbfb7.com/furnace.html}.
%
%\subsection*{\textit{Some notes and experiments on the electropermeabilization of viral membranes}}
%
%A 10-month attempt to follow up experimentally on previous research regarding the dielectric properties of viruses.\\ \\
%%, in the hopes of harnessing a phenomenon known as irreversible membrane electroporation, via optimized Brillouin precursors.
%Involved a custom, inexpensive synchronous photon-counting fluorescence DNA sampling system and a custom 12 GHz absorption spectrometer.
%%
%%Data was collected on the model organism bacteriophage T4, necessitating wet-lab techniques.\\ \\ Due to several missteps in design on my part, however, the study was thoroughly inconclusive. A report on the same is due to be edited.
%%
%
%Broad overview of skills gained:
%\begin{itemize}
%	\item \textbf{Documentation}: LaTeX, Jupyter notebooks, Reference management
%	\item \textbf{Software}: Data analysis and automation; Python, C++, Mathematica, with a smattering of Julia and MATLAB.
%	\item \textbf{Simulation}: Several dozen toolchains were in use, ranging from modified open-source electromagnetic simulation systems to molecular dynamics with GROMACS.
%	\item \textbf{Electronics}: Microwave electronics design, PCB design with KiCAD
%	\item \textbf{Fabrication}: Electronics prototyping, CNC mill and lathe operation, micromachining, microfluidics
%\end{itemize}
%
%
%
%\subsection*{R\&D in vacuum systems}
%
%A multi-year attempt to develop a high-current ion beam lithography system, involving vacuum chamber design and fabrication and several custom GPU-accelerated simulation tools written in CUDA/C++.
%
%
%\begin{tcolorbox}
%	\textbf{Personal:\\}
%	Member of SimCoLab hackerspace for 7 years.
%	Canadian and German citizenship.
%\end{tcolorbox}
%
%
%
%\end{multicols}
%
%
%\section{Gallery}
%\begin{figure}[H]
%	%	\makebox[\textwidth][c]{
%	\centering
%	\subfloat[]{
%		\includegraphics[width=0.5\textwidth]{vacuum}
%	}
%	\subfloat[]{
%		\includegraphics[width=0.5\textwidth]{digraph.png}
%	}
%	\caption*{Bespoke high vacuum system. GPU-accelerated multigrid data structure and electrostatics solver for particle-in-cell ion beam simulation.}
%	\hfill
%	
%\end{figure}
%
%
%
%\begin{figure}[H]
%	%	\makebox[\textwidth][c]{
%	\centering
%	\subfloat[]{
%		\includegraphics[width=0.4\textwidth]{image1}
%		
%	}
%	\subfloat[]{
%		\includegraphics[width=0.4\textwidth]{safesump.png}
%		
%	}
%	\caption*{Redundant controller with a 120 Mhz Atmel ARM processor, running \ntilde10k lines of high-reliability firmware.}
%	\hfill
%	
%\end{figure}
%
%
%
%\begin{figure}[H]
%	\centering
%	
%	\subfloat[]{
%		\includegraphics[width=0.5\textwidth]{pulse_exposure_setup}
%		
%	}
%	\subfloat[]{
%		\includegraphics[width=0.5\textwidth]{eppenwolf_2}
%	}
%	
%	\subfloat[]{
%		\includegraphics[width=0.4\textwidth]{x_gal.jpg}
%	}
%	\hfill
%	\subfloat[]{
%		\includegraphics[width=0.5\textwidth]{bronch_9GHz_500W_2}
%	}
%	
%	\caption*{\\ (a) A sub-nanosecond kilovolt pulse exposure cell (based on an off-the-shelf avalanche transistor pulser design). \\ (b) A 12 GHz microwave absorption spectrometer. \\(c) The very pretty opalescent blue culture caused by E. coli B trying to metabolize lactose in an indicator for the enzyme $\beta$-galactosidase.\\ (d) An FDTD simulation of electromagnetic interaction with tissue. }
%\end{figure}
%
%
%
%



\end{document}